\pdfpagewidth=8.5in \pdfpageheight=11in
\pdfhorigin=1in \pdfvorigin=1in

\input tugboat.sty

\title * {\ninebf UTF}-8 installations of {\ninebf CWEB} *
\author * Igor Liferenko *
\address * \tt igor.liferenko (at) gmail dot com *

\article

\head * Abstract *

We show how to implement \UTF-8 support in \acro{CWEB} [1]
by adding the arrays {\it xord\/} and {\it xchr\/}.
Immediately after reading a Unicode character from an input file, we
convert it to an 8-bit character using {\it xord\/}. On output the reverse
operation is done using {\it xchr\/}.
This allows us to leave core algorithms of \acro{CWEB} unchanged.

Incidentally, the described method allows to
{\font\tentex=cmtex10 \def\.#1{{\tt#1}}\def\\#1#2{`{\tentex\char'#1#2}'}%
use the extended character set [1] of \acro{CWEB}: the characters
\\13, \\01, \\31, \\32, \\34, \\35,
\\36, \\37, \\04, \\20, and \\21
can be typed as abbreviations for C~language digraphs `\.{++}', `\.{--}', `\.{->}',
`\.{!=}', `\.{<=}', `\.{>=}', `\.{==}', `\.{\char'174 \char'174 }', `\.{\&\&}',
`\.{<<}', and `\.{>>}',
respectively.}

\head * 1. Initialization *

(For brevity, in the diffs following, the original code in the
\acro{CWEB} source is preceded with |<| characters, and the
new code with |>|. Both are sometimes reformatted for presentation
in this article, and for readability we sometimes leave a blank line
between the pieces. The actual implementation uses the change files
|comm-utf8.ch|, |cweav-utf8.ch| and |ctang-utf8.ch|, together with
|common-utf8.ch| [2].){\emergencystretch=1em\hbadness=2100\par}
\smallskip
First, we add global arrays {\it xord\/} and {\it xchr\/} to |common.w| [1].
We declare the size of the {\it xord\/} array to be
$2^{16}$ bytes. This means that only values from the
basic multilingual plane (\acro{BMP}) of Unicode are permitted.
We use the |wchar_t| data type
for characters in input files
to accommodate Unicode values.

Background: this predefined C type allocates four bytes
per character (on most systems). Character constants of this type are
written as |L'...'|.

\verbatim
unsigned char xord[65536];
wchar_t xchr[256];
\endverbatim

\noindent
These same arrays must be used in |cweave.w| [1].
\verbatim
extern unsigned char xord[];
extern wchar_t xchr[];
\endverbatim

\noindent
In |ctangle.w| [1] only the {\it xchr\/} array is needed.
||extern wchar_t xchr[];||

We initialize the {\it xord\/} and {\it xchr\/} arrays
in the {\it common\_init\/} function of |common.w|.
%\medskip
%
First, in {\it xchr\/} we map all
\acro{ASCII} characters to themselves:
\verbatim
for (i=0; i<=0176; i++) xchr[i]=i;
\endverbatim
\medskip

Then we map the rest of the indexes of {\it xchr\/} to space.
\verbatim
for (i=0177; i<=0377; i++) xchr[i]=' ';
\endverbatim
%\medskip

Elements in the {\it xchr\/} array
are overridden using
the file |mapping.w| [2].
||@i mapping.w||
This file specifies the character(s)
required for a particular
installation of \acro{CWEB}, for example:
\smallskip
\indent{\font\tt=cmtt9 \tt xchr[0xf1] = L'\"e';}
\smallskip

The initialization of {\it xord\/} comes next.
All its indexes are mapped by default to 127, which is
the \acro{ASCII} character code ({\tt DEL}) that is prohibited in text files.
Then we make {\it xord\/} contain the inverse of the information in {\it xchr\/}.
\verbatim
for (i=0;i<=65535;i++) xord[i]=127;
for(i=0200; i<=0377; i++) xord[xchr[i]]=i;
for(i=0; i<=0176; i++) xord[xchr[i]]=i;
\endverbatim
%\medskip
It remains to set the |LC_CTYPE| locale category.
The behavior of the C~library functions used below depends on this value.
||setlocale(LC_CTYPE, "C.UTF-8");||
Finally, we need the necessary headers.
\verbatim
#include <wchar.h>
#include <locale.h>
\endverbatim

\head * 2. Input *

For automatic conversion from \UTF-8 to Unicode,
we change the {\it input\_ln\/} function to use
{\it fgetwc\/} [3] instead of {\it getc\/}.
Also, {\it ungetc\/} is changed to {\it ungetwc\/} [3]
and |EOF| must be replaced with |WEOF| [3] (for this, |int|
is changed to |wint_t| [3]).
\verbatim
< int c;
> wint_t c;

< while (k<=buffer_end && (c=getc(fp))
<   != EOF && c!='\n')
> while (k<=buffer_end && (c=fgetwc(fp))
>   != WEOF && c!=L'\n')

< if ((c=getc(fp))!=EOF && c!='\n') {
> if ((c=fgetwc(fp))!=WEOF && c!=L'\n') {

< ungetc(c,fp);
> ungetwc(c,fp);

< if (c==EOF && limit==buffer) return(0);
> if (c==WEOF && limit==buffer) return(0);
\endverbatim
\medskip
The conversion with {\it xord\/} is done immediately after a character
is read.
\verbatim
< if ((*(k++) = c) != ' ') limit = k;
> if ((*(k++) = xord[c]) != ' ') limit = k;
\endverbatim

\head * 3. Output *
We use {\it xchr\/} and {\it printf\/} with |%lc| conversion specifier
for characters, printed on terminal during error reporting.
\verbatim
< putchar(*k);
> printf("%lc",xchr[(unsigned char)*k]);
\endverbatim
\medskip
The {\it term\_write\/} macro uses the C~library function {\it fwrite\/} to output a range of
characters. We must use {\it xchr\/} for each character (except the newline character),
then convert it to \UTF-8
via {\it printf\/}, using |%lc| conversion specifier.
\verbatim
< @d term_write(a,b) fflush(stdout),
<      fwrite(a,sizeof(char),b,stdout)

> @d term_write(a,b) do { fflush(stdout);
>   for (int i = 0; i < b; i++)
>     if (*(a+i)=='\n') putchar('\n');
>     else printf("%lc",xchr[(unsigned char)
>       *(a+i)]); } while (0)
\endverbatim
\medskip

In |cweave.w| all output to files is done via the {\it c\_line\_write\/} macro.
This uses the C~library function {\it fwrite\/} to output a range of
characters. Since {\it xchr\/} must be used for each character,
we loop over this range and convert each character
to the external encoding and then to \UTF-8
via {\it fprintf}, using the |%lc| conversion specifier.
\verbatim
< fwrite(out_buf+1,sizeof(char),c,
<   active_file)

> for (int i = 0; i < c; i++)
>   fprintf(active_file, "%lc",
>     xchr[(eight_bits) *(out_buf+1+i)])
\endverbatim
\medskip
Similarly, in |ctangle.w|, before outputting characters in C~string constants,
convert each of them to the external encoding and then to \UTF-8 using
the |%lc| conversion specifier of {\it fprintf\/}.{\hfuzz=1.4pt\par}
\verbatim
< C_putc(a);
> fprintf(C_file,"%lc",xchr[(eight_bits)a]);
\endverbatim
\medskip
For other output code no special treatment is needed, since
all other output data is in \acro{ASCII}, which is part of \UTF-8
(except file names, which are already
in \UTF-8).

\head * 4. The file name buffer *
File names must be in \UTF-8. So, before appending characters to {\it cur\_file\_name\/},
we convert them to the external encoding and then to
\UTF-8 via C~library function {\it wctomb\/} [3].
\smallskip
\verbatim
< *k++=*loc++;

> { char mb[MB_CUR_MAX]; int len =
>   wctomb(mb,xchr[(unsigned char)*loc++]);
> if (k<=cur_file_name_end)
>   for (int i = 0; i<len; i++) *k++=mb[i];
> else k=cur_file_name_end+1; }
\endverbatim

\head * 5. Locale considerations *

|cweave.w| uses the locale-dependent C~library functions
\hbox{\it islower\/}, \hbox{\it isupper\/} and \hbox{\it tolower\/}
(the former two via {\it xislower\/} and {\it xisupper\/}
macros respectively).
But since we are assuming the \hbox{\UTF-8} locale, instead of these we must use
{\it iswlower\/}, {\it iswupper\/} and {\it towlower\/}
from |wctype.h| [3].
The trick is to convert from the internal encoding
to the external encoding before using these functions.
\verbatim
< xislower(*x)
> iswlower(xchr[(eight_bits)*p])

< xisupper(x)
> iswupper(xchr[(eight_bits) x])
\endverbatim
\medskip
For {\it towlower\/} the result must be converted
back from the external encoding to the internal
encoding.
\verbatim
< c=tolower(c)
> c=xord[towlower(xchr[(eight_bits)c])]
\endverbatim

\head * References *

\begingroup
\vskip-1pt

\frenchspacing
\parindent=0pt
\parskip=4pt plus4pt
\def\\{\hfil\break}
\everypar{\hangindent=1.5em\relax}
\raggedright

[1] Knuth, D. and Levy, S.
The CWEB System of Structured Documentation, 1993.
\\ISBN 0-201-57569-8

[2] Source of the present implementation.\\
{\tt https://github.com/igor-liferenko/cweb}

[3] Single Unix Specification. Introduction to ISO~C Amendment 1
(Multibyte Support Environment).\\
{\tt\catcode`_=11 https://unix.org/version2/whatsnew/\\login_mse.html}

\par\endgroup

\vskip 14pt
\advance\signaturewidth by 10pt
\makesignature
\endarticle
